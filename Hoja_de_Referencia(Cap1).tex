\documentclass{article}
\usepackage{graphicx} % Required for inserting images
\usepackage{algpseudocode}
\usepackage[spanish]{babel}
\usepackage{listings}
\usepackage{xcolor}
\graphicspath{{./Captura de pantalla de 2025-06-19 17-38-20/}}

\title{}
\author{Brigido Noguera}
\date{June 2025}

\begin{document}

\maketitle
\section{Prestaciones}

para maximizar las prestaciones se necesita minimizar el tiempo
de ejecucion.

\[  
    prestaciones_x = \frac{1}{tiempoDeEjecucion_x}
\]
    Asi, si las prestaciones de una maquina X son mayores a las de una maquina Y, se tiene:
\[
    prestaciones_x > prestaciones_y
\]
\[\frac{1}{tiempoDeEjecucion_x} > \frac{1}{tiempoDeEjecucion_y}\]

\[tiempoDeEjecucion_y = tiempoDeEjecucion_x\]

demuestra que X es mas rapido que Y.

\[\frac{prestaciones_x}{prestaciones_y}=n\]

"X es n veces mas rapida que Y"


\section{Prestaciones de CPU y sus factores}

\[
    tiempoDeEjecucionDeCPU = \frac{cicloDeRelojDeLaCPUparaElPrograma}{frecuenciaDeReloj(GHz)}
\]

\[
    ciclosDeRelojDeCPU = instruccionesDeUnPrograma * CPI
\]

\[
    tiepoDeCPU_A = ciclosDeCPU_A * tiempoDeCiclo_A
\]

\[
    tiempoDeEjecucion = \frac{numeroDeInstrucciones *CPI}{frecuenciaDeReloj}
\]

\[
    CPI = \frac{numeroDeCiclos(CPU)}{numeroDeInstrucciones}
\]


\section{Falacias y errores habituales(Ley de Amdahl)}

\[
    tiempoDe EjecucionDespuesDeLaMejora = \frac{tiempoDeEjecucionPorLaMejora}{cantidadDeMejora} + tiempoDeEjecucionNoAfectado
\]

\section{GCC (Compilador de C) y Optimizacion}

ejemplo del comando \verb|time|.

\begin{verbatim}
    gcc hola_mundo.c    // Compila el archivo
    time ./a.out        // Ejecuta y mide el su tiempo de ejecucion

    // Resultado //
    real	0m0,004s
    user	0m0,000s
    sys	0m0,004s

\end{verbatim}

Una de las funcionalidades más potentes del compilador es la de “optimizar el código”. Existen múltiples técnicas para analizar y transformar el código generado inicialmente por el compilador de forma que se obtenga un mejor rendimiento o en tiempo de ejecución, o en uso de memoria. Para que el compilador aplique estas técnicas se debe incluir la opción \verb|-O|. Compila de nuevo el programa con esta opción y ejecútalo de nuevo con el comando time como prefijo. Compara los dos tiempos de ejecución.

\begin{verbatim}
    gcc -O hola_mundo.c     // Compilamos con una optimizacion incluida (segun el codigo del programa)
\end{verbatim}

\end{document}